\section{Problem C}
\subsection{Task 1}
Lets define the vector $Z = [z_1,z_2,..,z_K]$ with the density function $f_Z(z_1,z_2,...,z_K)$. Since the $z_k$ values are independently drawn from a gamma distribution with parameter $\alpha_k$ , the following holds
\begin{align}
    f_Z(z_1,z_2,...,z_K) &= f_Z(z_1) f_Z(z_2) ... f_Z(z_K) \\
    &= \prod_{k=1}^K \frac{1}{\Gamma(\alpha_k)} z_k^{\alpha_k -1} e^{-z_k}
\end{align}

Lets define the vector
\begin{align}
    X &= [x_1, x_2, ... , x_{k-1}, v]\\
    &= g(Z)\\
    &= \left[ \frac{z_1}{\sum_{k=1}^K z_k}, ... ,\frac{z_{k-1}}{\sum_{k=1}^K z_k},  \sum_{k=1}^K z_k\right]
\end{align}

Using the change of variables formula the distribution of X can be found by the following

\begin{align}
    f_X(X) = f_z(g^{-1}(X)) \left| \frac{\partial}{\partial X} g^{-1}(X) \right| \label{PC:change_of_variable}
\end{align}

Firstly $g^{-1}(X)$ can be found as follows

\begin{align}
    Z &= g^{-1}(X)\\
    &= \left[ x_1 v, ... , x_{K-1} v, x_{K} v \right] \label{PC:g_inv_pre}
\end{align}

$x_{K}$ is undefined, but the sum over all $x_k$ are defined as 1, that is $\sum_{k=1}^K x_k=1$. The last term can therefor be expressed as $x_K = 1 - \sum_{k=1}^{K-1} x_k$. Inserting this into \ref{PC:g_inv_pre} gives
\begin{align}
    g^{-1}(x) = \left[ x_1 v, ... , x_{K-1} v, \left(1 - \sum_{k=1}^{K-1} x_k \right) v \right] \label{g_inv}
\end{align}

The Jacobin in expression \ref{PC:change_of_variable} can then be found as

\begin{align}
    \frac{\partial}{\partial X} g^{-1}(X) &= \begin{bmatrix} \frac{\partial g_1^{-1}}{\partial x_1} & \cdots & \frac{\partial g_K^{-1}}{\partial x_1}\\
    \vdots & & \vdots\\
    \frac{\partial g_1^{-1}}{\partial x_{k-1}} & \cdots & \frac{\partial g_K^{-1}}{\partial x_{k-1}} \\
    \frac{\partial g_1^{-1}}{\partial v} & \cdots & \frac{\partial g_K^{-1}}{\partial v}\end{bmatrix}\\
    &= \begin{bmatrix}
    v & 0 & 0 & \cdots & 0 & -v\\
    0 & v & 0 & \cdots & 0 & -v\\
    0 & 0 & v & \cdots & 0 & -v\\
    \vdots & \vdots & \vdots & & \vdots & \vdots\\
    0 & 0 & 0 & \cdots & v & -v\\
    x_1 & x_2 & x_3 & \cdots & x_{k-1} &\left(1 - \sum_{k=1}^{K-1} x_k \right)
    \end{bmatrix}
\end{align}

A sequence of matrix operations are done where each column (except for the last one) is subtracted from the last column.

\begin{align}
    \frac{\partial}{\partial X} g^{-1}(X) &= \begin{bmatrix}
    v & 0 & 0 & \cdots & 0 & 0\\
    0 & v & 0 & \cdots & 0 & 0\\
    0 & 0 & v & \cdots & 0 & 0\\
    \vdots & \vdots & \vdots & & \vdots & \vdots\\
    0 & 0 & 0 & \cdots & v & 0\\
    x_1 & x_2 & x_3 & \cdots & x_{k-1} & 1
    \end{bmatrix}
\end{align}

The determinant of this is simply
\begin{align}
    \left| \frac{\partial}{\partial X} g^{-1}(X) \right| = v^{k-1} \label{PC:det_of_jacobian}
\end{align}

Now the probability density function $f_X(X)$ can be evaluated. 

\begin{align}
    f_X(X) ={}&  f_X(x_1, x_2, ..., x_{k-1}, v) \label{PC:density_with_v}\\
    ={}& f_z(g^{-1}(X)) \left| \frac{\partial}{\partial X} g^{-1}(X) \right|\\
    ={}& \prod_{k=1}^{K-1} \frac{1}{\Gamma(\alpha_k)} g^{-1}(x_k)^{\alpha_k -1} e^{-g^{-1}(x_k)} \\
    {}& \frac{1}{\Gamma(\alpha_K)} g^{-1}(v)^{\alpha_K -1} e^{-g^{-1}(v)} v^{k-1}\\
    ={}& \frac{1}{\prod_{k=1}^K \Gamma(\alpha_k)} \prod_{k=1}^{K-1} \left( x_k^{\alpha_k-1} v^{\alpha_k-1} e^{-x_k v} \right)\\
    {}& \left( 1 - \sum_{k=1}^{K-1} x_k \right)^{\alpha_k-1} v^{\alpha_k-1} e^{-v \left( 1 - \sum_{k=1}^{K-1} x_k \right)}  v^{k-1}\\
    ={}& \frac{1}{\prod_{k=1}^K \Gamma(\alpha_k)} \prod_{k=1}^{K-1} \left( x_k^{\alpha_k-1} \right) \left( 1 - \sum_{k=1}^{K-1} x_k \right)^{\alpha_k-1} \\
    {}& \prod_{k=1}^{K} \left( c^{\alpha_k-1}\right) e^{-\sum_{k=0}^{K-1} (x_k v) - v \left( 1 - \sum_{k=1}^{K-1} x_k \right) }  v^{k-1}\\
    ={}& h(x_1,x_2,...,x_{k-1}) c^{\sum_{k=1}^{K}(\alpha_k-1)} e^{-v }  v^{k-1} \\
    ={}&h(x_1,x_2,...,x_{k-1}) c^{\sum_{k=1}^{K}(\alpha_k)-1} e^{-v}
\end{align}

To find the density function $f(x_1,...,x_{k-1})$, the function in \ref{PC:density_with_v} must be integrated over the entire domain of $v$. Since $v$ is a sum of gamma functions it cannot take negative values, making its domain from $-\inf$ to $\inf$.

\begin{align}
    f_X(x_1, x_2, ..., x_{k-1}) &= \int_{0}^{\inf} f_X(x_1, x_2, ..., x_{k-1},v) dv\\
    &= h(x_1,x_2,...,x_{k-1}) \int_{0}^{\inf} c^{\sum_{k=1}^{K}(\alpha_k)-1} e^{-v} dv
\end{align}

From the definition of the gamma function, this can be written as

\begin{align}
    f_X(x_1, x_2, ..., x_{k-1}) &= h(x_1,x_2,...,x_{k-1}) \Gamma(\sum_{k=1}^{K}\alpha_k) \\
    &= \frac{\Gamma(\sum_{k=1}^{K}\alpha_k)}{\prod_{k=1}^K \Gamma(\alpha_k)} \prod_{k=1}^{K-1} \left( x_k^{\alpha_k-1} \right) \left( 1 - \sum_{k=1}^{K-1} x_k \right)^{\alpha_k-1}
\end{align}

Which is the definition of the density function of the Dirchlet distribution given in the task. This means that the set of variables $(x_1 x_2 ... x_k)$ defined as $x_k = z_k / (z_1 + ... + z_K)$ are Dirchlet distributed when $(z_1, z_2, ... , z_K)$ are produced from a gamma distribution. The parameter vector in the Dirichlet distribution, $\mathbf{\alpha}$, is the ordered set of the $\alpha_k$ paramteres used in the gamma distributions.


\subsection{Task 2}

An R function that generates realisations from a Dirichlet distribution was implemented as follows, with the parameter vector $\alpha$ made randomly between 0 and 10. 


\begin{lstlisting}
n = 100000

K = 5;
alpha = runif(K,0,10)

x_vec = matrix(nrow=n,ncol=K)

for (i in 1:n){
  z = rgamma(K,alpha)
  v = sum(z)
  x = z/v
  x_vec[i,] = x
}

mean = colSums(x_vec)/n
offset =x_vec- t(t(rep(1,n)))%*%t(mean)
empirical_variance = colSums(offset^2)/n

expected_value = alpha/sum(alpha)
variance = alpha*(sum(alpha) -alpha)/(sum(alpha)^2 * (sum(alpha)+1)) 
\end{lstlisting}

To test if the implementation was correct, 100000 realizations were generated, and the mean and empirical variance was calculated. The theoretical expected value variance of a Dirichlet function is defined as 

\begin{align}
    E[x_i] &= \frac{alpha_i}{\sum_k \alpha_k}\\
    Var[x_i] &= \frac{\alpha_i (\alpha_0 - \alpha_i)}{\alpha_0^2 (\alpha_0 + 1)}\\
    \alpha_0 &= \sum_{i=1}^{K} \alpha_i
\end{align}

The test resulted in the empirical and theoretical variance and mean shown in table \ref{tab:PC:results}.

\begin{table}[h!]
    \centering
    \begin{tabular}{|c|c|c|c|c|c|}
    \hline
    &   $x_1$ & $x_2$ & $x_3$ & $x_4$ & $x_5$\\
    \hline
    empirical mean &  0.26116374 & 0.04357986 & 0.27295163 & 0.13570097 & 0.28660378\\
    expected value &  0.2609058 & 0.0436166 & 0.2726321  &0.1357966  &0.2870489\\
    mean deviation & 2.579310e-04 & -3.673436e-05  &3.195531e-04& -9.562904e-05 &-4.451208e-04 \\
    \hline
    empirical variance & 0.006243956 & 0.001334725 & 0.006403530 & 0.003779810 & 0.006592640 \\
    theoretical variance & 0.006198386 & 0.001340846 & 0.006374207 & 0.003772245 & 0.006578255\\
    variance deviation & 4.557055e-05 & -6.121071e-06  & 2.932295e-05  & 7.564649e-06 &  1.438518e-05\\
    \hline
    \end{tabular}
    \caption{Resulting comparison of theoretical and calculated values.}
    \label{tab:PC:results}
\end{table}
